\documentclass{bioinfo}
\usepackage{multirow}
\copyrightyear{2015} \pubyear{2015}

\access{Advance Access Publication Date: Day Month Year}
\appnotes{Manuscript Category}

\begin{document}
\firstpage{1}

\subtitle{Subject Section}

\title[short Title]{This is a title}
\author[Holt \textit{et~al}.]{James M. Holt\,$^{\text{\sfb 1,}*}$, Co-Author\,$^{\text{\sfb 2}}$ and Co-Author\,$^{\text{\sfb 2,}*}$}
\address{$^{\text{\sf 1}}$Department, Institution, City, Post Code, Country and \\
$^{\text{\sf 2}}$Department, Institution, City, Post Code,
Country.}

\corresp{$^\ast$To whom correspondence should be addressed.}

\history{Received on XXXXX; revised on XXXXX; accepted on XXXXX}

\editor{Associate Editor: XXXXXXX}

\abstract{\textbf{Motivation:} TODO.\\
\textbf{Results:} TODO. \\
\textbf{Availability:} TODO. \\
\textbf{Contact:} \href{jholt@hudsonalpha.org}{jholt@hudsonalpha.org}\\
\textbf{Supplementary information:} Supplementary data are available at \textit{Bioinformatics}
online.}

\maketitle

\section{Introduction}
WGS is currently being used as a molecular diagnostic tool for rare disease; blood draw + phenotypes are provided to analysis centers and these go through a standard alignment/variant calling pipeline

Because of size of data, variants are typically filtered down based on rare disease expectations; these variants are manually checked by two or more analysts and reported to clinical site

Manual curation is tedious and prone to error due to variant fatigue; thus tools that further reduce manual time on a case are paramount

%\enlargethispage{12pt}

\section{Approach}

Here we test methods that take phenotype, gene, transcript, and variant level annotations from the filtered variant list and predict whether a variant will ultimately be reported back to the clinical site.

The methods provide a boolean predictor and a ranking system.

\begin{methods}
\section{Methods}

\subsection{Variant annotation}
CODI does most of it; we use PyxisMap and HPOUtil for the rest

\subsection{Data cleaning}
Numerical data is given a number with out-of-bounds defaults

Categorical data is XXX (what do we decide to do with it?)

\subsection{Model training and tuning}

We tested a lot of models and tuned them

\end{methods}

\section{Discussion}
Results go here

\subsection{RENDERED DATA}
\begin{table*}
\centering
\begin{tabular}{c|c|c|c|c||c|c|c|c}
\input{data_rendered.tex}
\end{tabular}
\end{table*}

\subsection{blah blah}
Mathy stuff: Precision, recall, AUC, ROC, etc.

Practical stuff: in a case, this pushed reported variants to rank X out of Y (or something like this)

\section{Conclusion}

We did a thing

\section*{Acknowledgements}
TODO

Text Text Text Text Text Text  Text Text.  \citealp{Boffelli03} might want to know about  text
text text text\vspace*{-12pt}

\section*{Funding}
TODO

This work has been supported by the... Text Text  Text Text.\vspace*{-12pt}

%\bibliographystyle{natbib}
%\bibliographystyle{achemnat}
%\bibliographystyle{plainnat}
%\bibliographystyle{abbrv}
%\bibliographystyle{bioinformatics}
%
%\bibliographystyle{plain}
%
%\bibliography{Document}


\begin{thebibliography}{}

\bibitem[Bofelli {\it et~al}., 2000]{Boffelli03}
Bofelli,F., Name2, Name3 (2003) Article title, {\it Journal Name}, {\bf 199}, 133-154.

\bibitem[Bag {\it et~al}., 2001]{Bag01}
Bag,M., Name2, Name3 (2001) Article title, {\it Journal Name}, {\bf 99}, 33-54.

\bibitem[Yoo \textit{et~al}., 2003]{Yoo03}
Yoo,M.S. \textit{et~al}. (2003) Oxidative stress regulated genes
in nigral dopaminergic neurnol cell: correlation with the known
pathology in Parkinson's disease. \textit{Brain Res. Mol. Brain
Res.}, \textbf{110}(Suppl. 1), 76--84.

\bibitem[Lehmann, 1986]{Leh86}
Lehmann,E.L. (1986) Chapter title. \textit{Book Title}. Vol.~1, 2nd edn. Springer-Verlag, New York.

\bibitem[Crenshaw and Jones, 2003]{Cre03}
Crenshaw, B.,III, and Jones, W.B.,Jr (2003) The future of clinical
cancer management: one tumor, one chip. \textit{Bioinformatics},
doi:10.1093/bioinformatics/btn000.

\bibitem[Auhtor \textit{et~al}. (2000)]{Aut00}
Auhtor,A.B. \textit{et~al}. (2000) Chapter title. In Smith, A.C.
(ed.), \textit{Book Title}, 2nd edn. Publisher, Location, Vol. 1, pp.
???--???.

\bibitem[Bardet, 1920]{Bar20}
Bardet, G. (1920) Sur un syndrome d'obesite infantile avec
polydactylie et retinite pigmentaire (contribution a l'etude des
formes cliniques de l'obesite hypophysaire). PhD Thesis, name of
institution, Paris, France.

\end{thebibliography}
\end{document}
